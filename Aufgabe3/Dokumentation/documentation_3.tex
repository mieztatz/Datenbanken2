\documentclass{scrartcl}
\usepackage[ngerman]{babel}
\usepackage[utf8]{inputenc}
\usepackage{listingsutf8}
\usepackage{pdflscape}
\usepackage{color} 
\definecolor{listgray}{rgb}{0.88,0.88,0.88}

%\usepackage{lscape}
\KOMAoptions{titlepage=firstiscover}

\begin{document}
\lstset{language=SQL,
backgroundcolor=\color{listgray},
float=[htb],
frame=tbrl, %t: top, r, b, l 
  frameround=tttt,
  breaklines=true
} 
\begin{titlepage}
\titlehead{Hochschule München, Fakultät 07, SoSe 2016}
\subject{Datenbanken 2}
\title{Dokumentation zu Übung 3}
\subtitle{}
\author{Fabian Uhlmann \\Diana Irmscher}
\end{titlepage}

\maketitle

\section*{Aufgabe 2}
\begin{lstlisting}
--------------------------------------------------------
--  DDL for Function MIN_MAX_SCALE
--------------------------------------------------------

  CREATE OR REPLACE FUNCTION "MIN_MAX_SCALE" 
  (min_old NUMBER, min_new NUMBER, max_old NUMBER, max_new NUMBER, v NUMBER)
RETURN NUMBER
IS
BEGIN
  RETURN (((v - min_old)/(max_old - min_old))*(max_new - min_new)) + min_new;
END;

/
\end{lstlisting}
\begin{lstlisting}
--------------------------------------------------------
--  DDL for Procedure MIN_MAX_CALCULATOR
--------------------------------------------------------
-- Ergebnisse werden in neue Table eingetragen
  CREATE OR REPLACE PROCEDURE "MIN_MAX_CALCULATOR" 
  (min_new NUMBER,max_new NUMBER)
IS
min_old number;
BEGIN
 SELECT MIN(ZAHLEN) INTO min_old FROM NUMBERS;
  INSERT INTO NUMBERS_RESULT(
  SELECT MIN_MAX_SCALE(min_old, min_new, (SELECT MAX(ZAHLEN) FROM NUMBERS), max_new, ZAHLEN)
  FROM NUMBERS);
END;

/
\end{lstlisting}
\begin{lstlisting}
-- Alternative: Update in gleicher Table
CREATE OR REPLACE PROCEDURE "MIN_MAX_CALCULATOR" 
  (min_new NUMBER,max_new NUMBER)
IS
min_old number;
BEGIN
 SELECT MIN(ZAHLEN) INTO min_old FROM NUMBERS;
  UPDATE NUMBERS SET ZAHLEN = MIN_MAX_SCALE(min_old, min_new, (SELECT MAX(ZAHLEN) FROM NUMBERS), max_new, ZAHLEN);
END;
\end{lstlisting}

\begin{lstlisting}
--> Result: 
EXECUTE min_max_calculator(0,10);

SELECT * FROM NUMBERS ORDER BY ZAHLEN ASC;
\end{lstlisting}
\begin{tabular}{|r|}
\hline
 ZAHLEN \\
\hline
    5   \\
   10   \\
   20   \\
   25   \\
   42   \\
   50   \\
   53   \\
  100   \\
  120   \\
  142   \\
  242   \\
  250   \\
  342   \\
  350   \\
  420   \\
\hline
\end{tabular}


\section*{Aufgabe 3}
\begin{lstlisting}
--------------------------------------------------------
--  DDL for Table ANGESTELLTE
--------------------------------------------------------
  CREATE TABLE "ANGESTELLTE"(
	"A_NR" NUMBER(*,0),
	"A_NAME" VARCHAR2(50 BYTE), 
	"A_GEBURTSDATUM" DATE, 
	"A_BERUFSBEZEICHNUNG" VARCHAR2(60 BYTE), 
	"A_MONATSGEHALT" NUMBER(*,0), 
	"A_GESCHLECHT" VARCHAR2(10 BYTE), 
	PRIMARY KEY ("A_NR"));
	
\end{lstlisting}
\begin{lstlisting}
--------------------------------------------------------
--  DDL for Table ARBEITER
--------------------------------------------------------
  CREATE TABLE "ARBEITER"(	
	"A_NAME" VARCHAR2(30 BYTE), 
	"A_VORNAME" VARCHAR2(30 BYTE), 
	"A_GEBURTSMONAT" VARCHAR2(5 BYTE), 
	"A_STUNDENLOHN" NUMBER(*,0), 
	PRIMARY KEY ("A_NAME", "A_VORNAME"));
  
\end{lstlisting}
\begin{lstlisting}
--------------------------------------------------------
--  DDL for Table BERUFE
--------------------------------------------------------
  CREATE TABLE "BERUFE"(
	"B_CODE" NUMBER(*,0), 
	"B_TYPE" VARCHAR2(30 BYTE), 
	PRIMARY KEY ("B_CODE"));
\end{lstlisting}
\begin{lstlisting}
--------------------------------------------------------
--  DDL for Table GESCHLECHTER
--------------------------------------------------------
  CREATE TABLE "GESCHLECHTER"(
	"G_CODE" NUMBER(*,0), 
	"G_TYPE" VARCHAR2(10 BYTE), 
	PRIMARY KEY ("G_CODE"));
\end{lstlisting}
\begin{lstlisting}
--------------------------------------------------------
--  DDL for Table PERSONAL
--------------------------------------------------------
  CREATE TABLE "PERSONAL"(
	"P_NR" NUMBER(*,0), 
	"P_NAME" VARCHAR2(30 BYTE), 
	"P_VORNAME" VARCHAR2(30 BYTE), 
	"P_ALTER" NUMBER(*,0), 
	"P_GESCHLECHT" NUMBER(*,0), 
	"P_BERUFSCODE" NUMBER(*,0), 
	"P_JAHRESEINKOMMEN" NUMBER(*,0), 
	PRIMARY KEY ("P_NR"),
	FOREIGN KEY ("P_GESCHLECHT") REFERENCES "GESCHLECHTER" ("G_CODE"), 
	FOREIGN KEY ("P_BERUFSCODE") REFERENCES "BERUFE" ("B_CODE"));
\end{lstlisting}
\begin{lstlisting}
--------------------------------------------------------
--  DDL for Table ZUORDNUNG
--------------------------------------------------------
  CREATE TABLE "ZUORDNUNG"(
	"Z_NR" NUMBER(*,0), 
	"Z_TABLE_OLD" VARCHAR2(30 BYTE), 
	"Z_KEY_OLD" VARCHAR2(60 BYTE), 
	PRIMARY KEY ("Z_NR"),
	FOREIGN KEY ("Z_NR") REFERENCES "PERSONAL" ("P_NR"));
\end{lstlisting}
\begin{lstlisting}
--------------------------------------------------------
--  INSERTS for Table GESCHLECHTER
--------------------------------------------------------
Insert into GESCHLECHTER (G_CODE,G_TYPE) values ('0','unbekannt');
Insert into GESCHLECHTER (G_CODE,G_TYPE) values ('1','weiblich');
Insert into GESCHLECHTER (G_CODE,G_TYPE) values ('2','maennlich');
\end{lstlisting}

\end{document}