\documentclass{scrartcl}
\usepackage[ngerman]{babel}
\usepackage[utf8]{inputenc}
\usepackage{listingsutf8}
\usepackage{pdflscape}
\usepackage{color} 
\definecolor{listgray}{rgb}{0.88,0.88,0.88}

%\usepackage{lscape}
\KOMAoptions{titlepage=firstiscover}

\begin{document}
\lstset{language=SQL,
backgroundcolor=\color{listgray},
float=[htb],
frame=tbrl, %t: top, r, b, l 
  frameround=tttt,
  breaklines=true
} 
\begin{titlepage}
\titlehead{Hochschule München, Fakultät 07, SoSe 2016}
\subject{Datenbanken 2}
\title{Dokumentation zu Übung 3}
\subtitle{}
\author{Fabian Uhlmann \\Diana Irmscher}
\end{titlepage}

\maketitle

\section*{Aufgabe 2}
Realisierung einer Min-Max-Skalierung auf einem Attribut A einer Relation.
Skalierung selbst ist als Funktion implementiert, die mit den Parametern altes/neues Minimum/Maximum versehen ist,
einen Wert als Parameter entgegennimmt und den skalierten Wert zurückliefert.
Folgende Funktion wird berechnet:\\
$ v' = \frac{v - min_A}{max_A - min_A}\cdot (newMax_A - newMin_A) + newMin_A $
\begin{lstlisting}
--------------------------------------------------------
--  DDL for Function MIN_MAX_SCALE
--------------------------------------------------------

  CREATE OR REPLACE FUNCTION "MIN_MAX_SCALE" 
  (min_old NUMBER, min_new NUMBER, max_old NUMBER, max_new NUMBER, v NUMBER)
RETURN NUMBER
IS
BEGIN
  RETURN (((v - min_old)/(max_old - min_old))*(max_new - min_new)) + min_new;
END;

/
\end{lstlisting}
Die oben definierte Funktion wird von einer Prozedur aufgerufen werden, die für ein festes Attribut einer Relation zunächst den minimalen und den maximalen Wert der bisherigen
Attributwerte ermittelt und dann alle Werte durch die skalierten Werte ersetzt. 
\begin{lstlisting}
--------------------------------------------------------
--  DDL for Procedure MIN_MAX_CALCULATOR
--------------------------------------------------------
-- Ergebnisse werden in neue Table eingetragen
  CREATE OR REPLACE PROCEDURE "MIN_MAX_CALCULATOR" 
  (min_new NUMBER,max_new NUMBER)
IS
min_old number;
BEGIN
 SELECT MIN(ZAHLEN) INTO min_old FROM NUMBERS;
  INSERT INTO NUMBERS_RESULT(
  SELECT MIN_MAX_SCALE(min_old, min_new, (SELECT MAX(ZAHLEN) FROM NUMBERS), max_new, ZAHLEN)
  FROM NUMBERS);
END;

/
\end{lstlisting}
\begin{lstlisting}
-- Alternative: Update in gleicher Table
CREATE OR REPLACE PROCEDURE "MIN_MAX_CALCULATOR" 
  (min_new NUMBER,max_new NUMBER)
IS
min_old number;
BEGIN
 SELECT MIN(ZAHLEN) INTO min_old FROM NUMBERS;
  UPDATE NUMBERS SET ZAHLEN = MIN_MAX_SCALE(min_old, min_new, (SELECT MAX(ZAHLEN) FROM NUMBERS), max_new, ZAHLEN);
END;
\end{lstlisting}
Beispiel:
\begin{lstlisting}
--> Result: 
EXECUTE min_max_calculator(0,10);

SELECT * FROM NUMBERS ORDER BY ZAHLEN ASC;
\end{lstlisting}
\begin{tabular}{|r|}
\hline
 ZAHLEN \\
\hline
    5   \\
   10   \\
   20   \\
   25   \\
   42   \\
   50   \\
   53   \\
  100   \\
  120   \\
  142   \\
  242   \\
  250   \\
  342   \\
  350   \\
  420   \\
\hline
\end{tabular}

\newpage

\section*{Aufgabe 3}
Aus zwei verschiedenen Quellen werden zwei Tabellen Angestellte und Arbeiter
in den Arbeitsbereich eines Data Warehouse Systems geladen, die in eine
Tabelle Personal integriert werden sollen.\\
Die Tabellen besitzen die folgenden
Attribute:\\ \\
\begin{tabular}{l l}
Angestellte: & Name (String, Vorname und Nachname durch Blank
 getrennt; oder Nachname, Vorname) \\
 & Geburtsdatum (Format: Datum, YY/MM/DD) \\
 & Berufsbezeichnung \\
 & Monatsgehalt \\
 & Geschlecht: männlich bzw. weiblich \\
 & Angestelltennr (Primärschlüssel) \\
\end{tabular}

\begin{tabular}{l l}
Arbeiter: & Name \\
& Vorname \\
& Geburtsmonat (Format: String MM.YY) \\
& Stundenlohn \\
\end{tabular}

\begin{tabular}{l l}
Personal: & Personalnr. (Primärschlüssel) \\
 & Name \\
 & Vorname \\
 & Alter \\
 & Geschlecht: (0 (unbekannt), 1 (weiblich), 2 (männlich)) \\
 & Berufscode \\
 & Jahreseinkommen \\
\end{tabular}
\\
Die Tabelle Personal enthält einen neuen (generierten) Schlüssel. 
\begin{lstlisting}
--------------------------------------------------------
--  DDL for Sequence PNR_SEQUENCE
--------------------------------------------------------
   CREATE SEQUENCE  "PNR_SEQUENCE";

/
\end{lstlisting}

\begin{lstlisting}
--------------------------------------------------------
--  DDL for Table ANGESTELLTE
--------------------------------------------------------
  CREATE TABLE "ANGESTELLTE" 
   (	"A_NR" NUMBER, 
	"A_NAME" VARCHAR2(50), 
	"A_GEBURTSDATUM" DATE, 
	"A_BERUFSBEZEICHNUNG" VARCHAR2(60), 
	"A_MONATSGEHALT" NUMBER, 
	"A_GESCHLECHT" VARCHAR2(10),
	PRIMARY KEY ("A_NR")
   );

/
\end{lstlisting}

\begin{lstlisting}
--------------------------------------------------------
--  DDL for Table ARBEITER
--------------------------------------------------------
  CREATE TABLE "ARBEITER" 
   (	"A_NAME" VARCHAR2(30), 
	"A_VORNAME" VARCHAR2(30), 
	"A_GEBURTSMONAT" VARCHAR2(5), 
	"A_STUNDENLOHN" NUMBER,
	PRIMARY KEY ("A_NAME", "A_VORNAME")
   );

/
\end{lstlisting}
Für die Berufscodes ist eine Code-Tabelle zu definieren, die Berufsbezeichnungen
(einschließlich der Bezeichnung Arbeiter) einen Code zuordnet.
\begin{lstlisting}
--------------------------------------------------------
--  DDL for Table BERUFE
--------------------------------------------------------
  CREATE TABLE "BERUFE" 
   (	"B_CODE" NUMBER, 
	"B_TYPE" VARCHAR2(30),
	PRIMARY KEY ("B_CODE")
   );

/
\end{lstlisting}
Für die „Geschlechtsbestimmung“ der Arbeiter verwenden Sie bitte eine Hilfstabelle, die
Vornamen jeweils ein oder mehre Geschlechter zuordnen. Ergibt der Abgleich mit
dieser Tabelle kein eindeutiges Ergebnis wird der Eintrag in der Zieltabelle auf 0
(unbekannt) gesetzt. 
\begin{lstlisting}
--------------------------------------------------------
--  DDL for Table GESCHLECHTER
--------------------------------------------------------
  CREATE TABLE "GESCHLECHTER" 
   (	"G_NAME" VARCHAR2(15), 
	"G_CODE" NUMBER,
	PRIMARY KEY ("G_NAME")
   );

/
\end{lstlisting}

\begin{lstlisting}
--------------------------------------------------------
--  DDL for Table PERSONAL
--------------------------------------------------------
  CREATE TABLE "PERSONAL" 
   (	"P_NR" NUMBER, 
	"P_NAME" VARCHAR2(30), 
	"P_VORNAME" VARCHAR2(30), 
	"P_ALTER" NUMBER, 
	"P_GESCHLECHT" NUMBER, 
	"P_BERUFSCODE" NUMBER, 
	"P_JAHRESEINKOMMEN" NUMBER,
	PRIMARY KEY ("P_NR"),
	FOREIGN KEY ("P_BERUFSCODE") REFERENCES "BERUFE" ("B_CODE")
   );

/
\end{lstlisting}
Für die Beziehung
zwischen diesem und den Quelltabellen soll eine Zuordnungstabelle verwaltet
werden.
\begin{lstlisting}
--------------------------------------------------------
--  DDL for Table ZUORDNUNG
--------------------------------------------------------

  CREATE TABLE "ZUORDNUNG" 
   (	"Z_NR" NUMBER, 
	"Z_TABLE_OLD" VARCHAR2(30), 
	"Z_KEY_OLD" VARCHAR2(60),
	PRIMARY KEY ("Z_NR"),
	FOREIGN KEY ("Z_NR") REFERENCES "PERSONAL" ("P_NR")
   );

/
\end{lstlisting}
Programm, das die Integration ausführt und sowohl für
den initialen Ladevorgang der Tabelle Personal als auch für deren Fortschreibung
geeignet ist (d.h. Angestellte und Arbeiter enthalten jeweils nur neue bzw. geänderte
Datensätze).
\begin{lstlisting}
--------------------------------------------------------
--  DDL for Function GETAGE_DATE
--------------------------------------------------------
  CREATE OR REPLACE FUNCTION "GETAGE_DATE" 
  (birthdate Date)
RETURN VARCHAR2
IS
BEGIN
  RETURN Trunc((months_between(sysdate, birthdate) /12),0);
END;

/
\end{lstlisting}

\begin{lstlisting}
--------------------------------------------------------
--  DDL for Function GETAGE_STRING
--------------------------------------------------------
  CREATE OR REPLACE FUNCTION "GETAGE_STRING" 
  (birthdate VARCHAR)
RETURN VARCHAR2
age DATE;
BEGIN
  -- SELECT EXTRACT(MONTH FROM SYSDATE) FROM DUAL;
  SELECT TO_DATE(birthdate, 'MM.RR') INTO age FROM DUAL;
  RETURN Trunc((months_between(sysdate, age) /12),0);
END;

/
\end{lstlisting}

\begin{lstlisting}
--------------------------------------------------------
--  DDL for Function GETFIRSTNAME
--------------------------------------------------------
  CREATE OR REPLACE FUNCTION "GETFIRSTNAME" 
  (fname VARCHAR2)
RETURN VARCHAR2
IS
BEGIN
  RETURN SUBSTR(fname,0, instr(fname,' ')-1);
END;

/
\end{lstlisting}

\begin{lstlisting}
--------------------------------------------------------
--  DDL for Function GETGENDERCODE
--------------------------------------------------------
  CREATE OR REPLACE FUNCTION "GETGENDERCODE" 
  (gender VARCHAR2, firstname VARCHAR2)
RETURN NUMBER
CURSOR CGCODE IS
	SELECT G_CODE
	FROM Geschlechter
  WHERE G_NAME = firstname;
gendercode NUMBER;
tmp NUMBER;
BEGIN
  CASE gender
    WHEN 'maennlich' THEN gendercode := 2;
    WHEN 'weiblich' THEN gendercode := 1;
    ELSE gendercode := 0;
  END CASE;

  OPEN CGCODE;
  FETCH CGCODE into tmp;
  IF CGCODE%NOTFOUND THEN
    INSERT INTO GESCHLECHTER (G_NAME, G_CODE) VALUES (firstname,gendercode);
  ELSE 
    IF gendercode != 0 AND gendercode != tmp THEN
      UPDATE GESCHLECHTER SET G_CODE = gendercode WHERE G_NAME = firstname;
    ELSE gendercode := tmp;
    END IF;
  END IF
  RETURN gendercode;
END;

/
\end{lstlisting}

\begin{lstlisting}
--------------------------------------------------------
--  DDL for Function GETLASTNAME
--------------------------------------------------------
  CREATE OR REPLACE FUNCTION "DBST47"."GETLASTNAME" 
  (lname VARCHAR2)
RETURN VARCHAR2
IS
BEGIN
  RETURN SUBSTR(lname,INSTR(lname,' ')+1);
END;

/
\end{lstlisting}

\begin{lstlisting}
--------------------------------------------------------
--  DDL for Function GETJOBCODE
--------------------------------------------------------
  CREATE OR REPLACE FUNCTION "GETJOBCODE" 
  (jobname VARCHAR2)
RETURN NUMBER
IS
CURSOR CBCODE IS
	SELECT B_CODE
	FROM Berufe
	WHERE B_TYPE = jobname;
jobcode NUMBER;
BEGIN
  OPEN CBCODE;
  FETCH CBCODE into jobcode;
  IF CBCODE%NOTFOUND THEN 
    SELECT max(B_CODE) INTO jobcode FROM BERUFE; 
    IF jobcode IS NULL THEN jobcode := 0;
    ELSE jobcode := jobcode + 1;
    END IF;
    INSERT INTO BERUFE (B_CODE, B_TYPE) VALUES (jobcode,jobname); 
  END IF;
  RETURN jobcode;
END;

/
\end{lstlisting}

\begin{lstlisting}
--------------------------------------------------------
--  DDL for Function GETMONEY
--------------------------------------------------------
  CREATE OR REPLACE FUNCTION "GETMONEY" 
  (monthmoney NUMBER)
RETURN NUMBER
IS
BEGIN
  RETURN (monthmoney * 12);
END;

/
\end{lstlisting}

\begin{lstlisting}
--------------------------------------------------------
--  DDL for Procedure TRANSFORMATION_ANGESTELLTE
--------------------------------------------------------
  CREATE OR REPLACE PROCEDURE "TRANSFORMATION_ANGESTELLTE" 
IS
a_nr NUMBER;
p_nr NUMBER;
p_name VARCHAR2(30);
p_vorname VARCHAR2(30);
p_age DATE;
p_geschlecht VARCHAR2(10);
p_job VARCHAR(50);
p_money NUMBER;
CURSOR CANGST IS
	SELECT A_Nr, A_Name, A_Geburtsdatum, A_Berufsbezeichnung, A_Monatsgehalt, A_Geschlecht
	FROM Angestellte;
BEGIN
  OPEN CANGST;
  LOOP 
	FETCH CANGST INTO a_nr, p_name, p_age, p_job, p_money, p_geschlecht;
	EXIT WHEN CANGST%NOTFOUND;
	SELECT pnr_sequence.nextval INTO p_nr FROM DUAL;
  	SELECT GETFIRSTNAME(p_name) INTO p_vorname FROM DUAL;
  	INSERT INTO PERSONAL(p_nr,p_name,p_vorname,p_alter,p_geschlecht,p_berufscode,p_jahreseinkommen) VALUES (p_nr,GETLASTNAME(p_name),p_vorname,GETAGE_DATE(p_age),GETGENDERCODE(p_geschlecht,p_vorname),GETJOBCODE(p_job),GETMONEY(p_money));
	INSERT INTO ZUORDNUNG (Z_NR, Z_TABLE_OLD, Z_KEY_OLD) VALUES (p_nr, 'Angestellter', TO_CHAR(a_nr, '99999999'));
  END LOOP; 
  CLOSE CANGST;
END;

/
\end{lstlisting}

\begin{lstlisting}
--------------------------------------------------------
--  DDL for Procedure TRANSFORMATION_ARBEITER
--------------------------------------------------------
  CREATE OR REPLACE PROCEDURE "TRANSFORMATION_ARBEITER" 
IS
p_nr NUMBER;
p_name VARCHAR2(30);
p_vorname VARCHAR2(30);
p_age VARCHAR2(5);
p_geschlecht VARCHAR2(10);
p_job VARCHAR(50);
p_money NUMBER;
arb_nr VARCHAR2(60);
CURSOR CARB IS
	SELECT A_Name, A_Vorname, A_Geburtsmonat, A_Stundenlohn
	FROM Arbeiter;
BEGIN
  OPEN CARB;
  LOOP 
	FETCH CARB INTO p_name, p_vorname, p_age, p_money;
	EXIT WHEN CARB%NOTFOUND;
	SELECT pnr_sequence.nextval INTO p_nr FROM DUAL;
 	INSERT INTO PERSONAL(p_nr,p_name,p_vorname,p_alter,p_geschlecht,p_berufscode,p_jahreseinkommen) VALUES (p_nr,p_name,p_vorname,GETAGE_STRING(p_age),GETGENDERCODE('unbekannt',p_vorname),GETJOBCODE('Arbeiter'),GETMONEY(p_money*4*40));
	arb_nr := CONCAT(CONCAT(p_name,','),p_vorname);
	INSERT INTO ZUORDNUNG (Z_NR, Z_TABLE_OLD, Z_KEY_OLD) VALUES (p_nr, 'Arbeiter', arb_nr);
  END LOOP; 
  CLOSE CARB;
END;

/
\end{lstlisting}

\begin{lstlisting}
--------------------------------------------------------
--  DDL for Trigger UPDATE_ARBEITER
--------------------------------------------------------
CREATE OR REPLACE TRIGGER UPDATE_ARBEITER
  AFTER 
    INSERT OR 
    UPDATE OR 
    DELETE 
  ON Arbeiter
  FOR EACH ROW
DECLARE
  z_nr NUMBER;
  arb_nr VARCHAR2(60);
  p_nr NUMBER;
BEGIN
   IF INSERTING THEN
    SELECT pnr_sequence.nextval INTO p_nr FROM DUAL;
    INSERT INTO PERSONAL(p_nr,p_name,p_vorname,p_alter,p_geschlecht,p_berufscode,p_jahreseinkommen) VALUES (p_nr,:NEW.A_NAME,:NEW.A_VORNAME,GETAGE_STRING(:NEW.A_GEBURTSMONAT),GETGENDERCODE('unbekannt',:NEW.A_VORNAME),GETJOBCODE('Arbeiter'),GETMONEY(:NEW.A_STUNDENLOHN*4*40));
    arb_nr := CONCAT(CONCAT(:NEW.A_NAME,','),:NEW.A_VORNAME);
    INSERT INTO ZUORDNUNG (Z_NR, Z_TABLE_OLD, Z_KEY_OLD) VALUES (p_nr, 'Arbeiter', arb_nr);
    
   ELSIF UPDATING THEN
    IF :OLD.A_NAME != :NEW.A_NAME THEN
      SELECT z.Z_NR INTO z_nr FROM ZUORDNUNG z WHERE z.Z_KEY_OLD = CONCAT(CONCAT(:OLD.A_NAME,','),:OLD.A_VORNAME); /* --> dieses SELECT ggf. auslagern und direkt nach ELSIF UPDATING, da es wird in allen IFs von Updating benoetigt */
      UPDATE PERSONAL p SET p.P_NAME = :NEW.A_NAME WHERE p.P_NR = z_nr;
      SELECT p.P_NR INTO p_nr FROM PERSONAL p WHERE p.P_NAME = :NEW.A_NAME AND p.P_VORNAME = :OLD.A_VORNAME; /* notwendig, da sonst die Komplette Spalte Z_KEY_OLD in Table ZUORDNUNG mit geaendertem Namen ueberschrieben wird*/
      arb_nr := CONCAT(CONCAT(:NEW.A_NAME,','),:OLD.A_VORNAME);
      UPDATE ZUORDNUNG z SET z.Z_KEY_OLD = arb_nr WHERE z.z_nr = p_nr;
    END IF;
    
     IF :OLD.A_VORNAME != :NEW.A_VORNAME THEN
      SELECT z.Z_NR INTO z_nr FROM ZUORDNUNG z WHERE z.Z_KEY_OLD = CONCAT(CONCAT(:OLD.A_NAME,','),:OLD.A_VORNAME);
      /* DBMS_OUTPUT.PUT_LINE(z_nr); -- Fuer Debugging = Ausgabe auf DBMS-Console */
      UPDATE PERSONAL p SET p.P_VORNAME = :NEW.A_VORNAME WHERE p.P_NR = z_nr;
      SELECT p.P_NR INTO p_nr FROM PERSONAL p WHERE p.P_NAME = :OLD.A_NAME AND p.P_VORNAME = :NEW.A_VORNAME; /* notwendig, da sonst die Komplette Spalte Z_KEY_OLD in Table ZUORDNUNG mit geaendertem Namen ueberschrieben wird*/
      arb_nr := CONCAT(CONCAT(:OLD.A_NAME,','),:NEW.A_VORNAME);
      UPDATE ZUORDNUNG zg SET zg.Z_KEY_OLD = arb_nr WHERE zg.z_nr = p_nr;
    END IF;
    
    IF :OLD.A_GEBURTSMONAT != :NEW.A_GEBURTSMONAT THEN
      SELECT z.Z_NR INTO z_nr FROM ZUORDNUNG z WHERE z.Z_KEY_OLD = CONCAT(CONCAT(:OLD.A_NAME,','),:OLD.A_VORNAME);
      UPDATE PERSONAL p SET p.p_ALTER = GETAGE_STRING(:NEW.A_GEBURTSMONAT) WHERE p.P_NR = z_nr;
    END IF;

    IF :OLD.A_STUNDENLOHN != :NEW.A_STUNDENLOHN THEN
      SELECT z.Z_NR INTO z_nr FROM ZUORDNUNG z WHERE z.Z_KEY_OLD = CONCAT(CONCAT(:OLD.A_NAME,','),:OLD.A_VORNAME);
      UPDATE PERSONAL p SET p.P_JAHRESEINKOMMEN = GETMONEY(:NEW.A_STUNDENLOHN * 40 *4) WHERE p.P_NR = z_nr;
    END IF;
    
   ELSIF DELETING THEN
      SELECT z.Z_NR INTO z_nr FROM ZUORDNUNG z WHERE z.Z_KEY_OLD = CONCAT(CONCAT(:OLD.A_NAME,','),:OLD.A_VORNAME);
      DELETE FROM ZUORDNUNG z WHERE z.Z_NR = z_nr AND z.Z_KEY_OLD = CONCAT(CONCAT(:OLD.A_NAME,','),:OLD.A_VORNAME);
      DELETE FROM PERSONAL p WHERE p.P_NR = z_nr;
    
   ELSE NULL;
   END IF;
END;

/
\end{lstlisting}

\begin{lstlisting}
--------------------------------------------------------
--  DDL for Trigger UPDATE_ANGESTELLTE
--------------------------------------------------------
CREATE OR REPLACE TRIGGER UPDATE_ANGESTELLTE
  AFTER 
    INSERT OR 
    UPDATE OR 
    DELETE 
  ON ANGESTELLTE
  FOR EACH ROW
DECLARE
  z_nr NUMBER;
  arb_nr VARCHAR(60);
  p_nr NUMBER;
  p_vorname VARCHAR2(30);
BEGIN
   IF INSERTING THEN
    SELECT pnr_sequence.nextval INTO p_nr FROM DUAL;
    SELECT GETFIRSTNAME(:NEW.A_NAME) INTO p_vorname FROM DUAL;
    INSERT INTO PERSONAL(p_nr,p_name,p_vorname,p_alter,p_geschlecht,p_berufscode,p_jahreseinkommen) VALUES (p_nr,GETLASTNAME(:NEW.A_NAME),p_vorname,GETAGE_DATE(:NEW.A_GEBURTSDATUM),GETGENDERCODE(:NEW.A_GESCHLECHT,p_vorname),GETJOBCODE(:NEW.A_BERUFSBEZEICHNUNG),GETMONEY(:NEW.A_MONATSGEHALT));
    INSERT INTO ZUORDNUNG (Z_NR, Z_TABLE_OLD, Z_KEY_OLD) VALUES (p_nr, 'Angestellter', TO_CHAR(:NEW.A_NR, '99999999'));
    
   ELSIF UPDATING THEN

     SELECT z.Z_NR INTO z_nr FROM ZUORDNUNG z WHERE z.Z_KEY_OLD = TO_CHAR(:OLD.A_NR, '99999999');

     IF :OLD.A_NR != :NEW.A_NR THEN
      SELECT p.P_NR INTO p_nr FROM PERSONAL p WHERE p.P_NR = z_nr AND p.P_NAME = GETLASTNAME(:OLD.A_NAME) AND p.P_VORNAME = GETFIRSTNAME(:OLD.A_NAME); 
      UPDATE ZUORDNUNG z SET z.Z_KEY_OLD = TO_CHAR(:NEW.A_NR, '99999999') WHERE z.Z_NR = p_nr;
     END IF;

     IF :OLD.A_NAME != :NEW.A_NAME THEN
      UPDATE PERSONAL p SET p.P_NAME = GETLASTNAME(:NEW.A_NAME), p.P_VORNAME = GETFIRSTNAME(:NEW.A_NAME) WHERE p.P_NR = z_nr;
     END IF;
     
     IF :OLD.A_GEBURTSDATUM != :NEW.A_GEBURTSDATUM THEN
      UPDATE PERSONAL p SET p.p_ALTER = GETAGE_DATE(:NEW.A_GEBURTSDATUM) WHERE p.P_NR = z_nr;
     END IF;
    
     IF :OLD.A_BERUFSBEZEICHNUNG != :NEW.A_BERUFSBEZEICHNUNG THEN
      UPDATE PERSONAL p SET p.p_BERUFSCODE = GETJOBCODE(:NEW.A_BERUFSBEZEICHNUNG) WHERE p.P_NR = z_nr;
     END IF;

     IF :OLD.A_MONATSGEHALT != :NEW.A_MONATSGEHALT THEN
      UPDATE PERSONAL p SET p.P_JAHRESEINKOMMEN = GETMONEY(:NEW.A_MONATSGEHALT) WHERE p.P_NR = z_nr;
     END IF;

     IF :OLD.A_GESCHLECHT != :NEW.A_GESCHLECHT THEN
      UPDATE PERSONAL p SET p.P_geschlecht = GETGENDERCODE(:NEW.A_GESCHLECHT,GETFIRSTNAME(:OLD.A_NAME)) WHERE p.P_NR = z_nr;
     END IF;
    
   ELSIF DELETING THEN
      SELECT z.Z_NR INTO z_nr FROM ZUORDNUNG z WHERE z.Z_KEY_OLD = TO_CHAR(:OLD.A_NR, '99999999');
      DELETE FROM ZUORDNUNG z WHERE z.Z_NR = z_nr AND z.Z_KEY_OLD = TO_CHAR(:OLD.A_NR, '99999999');
      DELETE FROM PERSONAL p WHERE p.P_NR = z_nr;
    
   ELSE NULL;
   END IF;
END;

/
\end{lstlisting}

\begin{lstlisting}
--------------------------------------------------------
--  Inserts in Table ANGESTELLTE
--------------------------------------------------------
DELETE FROM ANGESTELLTE;
Insert into ANGESTELLTE (A_NR,A_NAME,A_GEBURTSDATUM,A_BERUFSBEZEICHNUNG,A_MONATSGEHALT,A_GESCHLECHT) values ('1','Fabian Uhlmann',to_date('03.11.88','DD.MM.RR'),'Informatiker','2000','maennlich');
Insert into ANGESTELLTE (A_NR,A_NAME,A_GEBURTSDATUM,A_BERUFSBEZEICHNUNG,A_MONATSGEHALT,A_GESCHLECHT) values ('2','Diana Irmscher',to_date('01.01.90','DD.MM.RR'),'Informatiker','2001','weiblich');
Insert into ANGESTELLTE (A_NR,A_NAME,A_GEBURTSDATUM,A_BERUFSBEZEICHNUNG,A_MONATSGEHALT,A_GESCHLECHT) values ('3','Alexandra Vogel',to_date('01.10.92','DD.MM.RR'),'Informatiker','9999','weiblich');
Insert into ANGESTELLTE (A_NR,A_NAME,A_GEBURTSDATUM,A_BERUFSBEZEICHNUNG,A_MONATSGEHALT,A_GESCHLECHT) values ('4','Alexander Boxhorn',to_date('27.07.82','DD.MM.RR'),'Logistiker','1375','maennlich');

/
\end{lstlisting}

\begin{lstlisting}
--------------------------------------------------------
--  Inserts in Table ARBEITER
--------------------------------------------------------
DELETE FROM ARBEITER;
Insert into ARBEITER (A_NAME,A_VORNAME,A_GEBURTSMONAT,A_STUNDENLOHN) values ('Meister','Bob','11.88',20);
Insert into ARBEITER (A_NAME,A_VORNAME,A_GEBURTSMONAT,A_STUNDENLOHN) values ('Mueller','Sarah','07.95',10);
Insert into ARBEITER (A_NAME,A_VORNAME,A_GEBURTSMONAT,A_STUNDENLOHN) values ('Bach','Hans','01.75',5);
Insert into ARBEITER (A_NAME,A_VORNAME,A_GEBURTSMONAT,A_STUNDENLOHN) values ('Heinz','Karl','11.88',8.5);

/
\end{lstlisting}

\begin{lstlisting}
--------------------------------------------------------
--  Inserts in Table GESCHLECHTER
--------------------------------------------------------
DELETE FROM GESCHLECHTER;
Insert into GESCHLECHTER (G_NAME,G_CODE) values ('Alexandra','1');
Insert into GESCHLECHTER (G_NAME,G_CODE) values ('Fabian','2');

/
\end{lstlisting}


Nun werden Änderungen in die Tabellen eingetragen.
\begin{lstlisting}
--------------------------------------------------------
--  Testcases
--------------------------------------------------------
DELETE FROM ZUORDNUNG;
DELETE FROM PERSONAL;
/*1*/ EXECUTE TRANSFORMATION_ARBEITER;
/*2*/ EXECUTE TRANSFORMATION_ANGESTELLTE;

/*3*/ Insert into ARBEITER (A_NAME,A_VORNAME,A_GEBURTSMONAT,A_STUNDENLOHN) values ('Kapitaen','Blaubaer','05.44',33); 
/*4*/ UPDATE ARBEITER SET A_NAME = 'Meyer' WHERE A_NAME = 'Meister'; /* !!! Table Zuordnung darf/sollte nur im einen Datensatz das Attribut Z_KEY_OLD updaten */
/*5*/ UPDATE ARBEITER SET A_VORNAME = 'Hans-Joachim' WHERE A_NAME = 'Bach'; /* !!! Table Zuordnung darf/sollte nur im einen Datensatz das Attribut Z_KEY_OLD updaten */
/*6*/ UPDATE ARBEITER SET A_GEBURTSMONAT = '01.01' WHERE A_NAME = 'Heinz';
/*7*/ UPDATE ARBEITER SET A_STUNDENLOHN = 3.5 WHERE A_NAME = 'Mueller';
/*8*/ DELETE FROM ARBEITER WHERE A_NAME = 'Kapitaen'; /* In Table Zuordnung darf/soll nur 1 Datensatz entfernt werden */

/*9*/ Insert into ANGESTELLTE (A_NR,A_NAME,A_GEBURTSDATUM,A_BERUFSBEZEICHNUNG,A_MONATSGEHALT,A_GESCHLECHT) values ('5','Max Mustermann',to_date('10.03.67','DD.MM.RR'),'BWL','850','maennlich');
/*10*/ UPDATE ANGESTELLTE SET A_NR = '10' WHERE A_NAME = 'Diana Irmscher'; /* !!! Table Zuordnung darf/sollte nur im einen Datensatz das Attribut Z_KEY_OLD updaten */
/*11*/ UPDATE ANGESTELLTE SET A_NAME = 'Fabius Uhlmex' WHERE A_NAME = 'Fabian Uhlmann';
/*12*/ UPDATE ANGESTELLTE SET A_GEBURTSDATUM = 03.10.1955 WHERE A_NAME = 'Alexandra Vogel';
/*13*/ UPDATE ANGESTELLTE SET A_BERUFSBEZEICHNUNG = 'Facility Management' WHERE A_NAME = 'Alexander Boxhorn';
/*14*/ UPDATE ANGESTELLTE SET A_MONATSGEHALT = 777 WHERE A_NAME = 'Alexander Boxhorn';
/*15*/ UPDATE ANGESTELLTE SET A_GESCHLECHT = 'weiblich' WHERE A_NAME = 'Max Mustermann';
/*16*/ DELETE FROM ANGESTELLTE WHERE A_NAME = 'Max Mustermann'; /* !!! Table Zuordnung darf/sollte nur im einen Datensatz das Attribut Z_KEY_OLD updaten */

\end{lstlisting}

\end{document}