\documentclass{scrartcl}
\usepackage[ngerman]{babel}
\usepackage[utf8]{inputenc}
\usepackage{listingsutf8}
\usepackage{pdflscape}

%\usepackage{lscape}
\KOMAoptions{titlepage=firstiscover}

\begin{document}
\lstset{language=SQL} 
\begin{titlepage}
\titlehead{Hochschule München, Fakultät 07, SoSe 2016}
\subject{Datenbanken 2}
\title{Dokumentation zu Übung 2}
\subtitle{}
\author{Fabian Uhlmann \\Diana Irmscher}
\end{titlepage}

\maketitle

\section*{Aufgabe 2}
\subsection*{1. Möglichkeit}
\begin{lstlisting}
--------------------------------------------------------
--  DDL for Type ADRESSE_T
--------------------------------------------------------
CREATE OR REPLACE TYPE "ADRESSE_T" AS OBJECT(
Strasse  VARCHAR2(30),
HNR INTEGER,
PLZ INTEGER,
Ort VARCHAR2(20)) NOT FINAL;
\end{lstlisting}
\begin{lstlisting}
/
--------------------------------------------------------
--  DDL for Type KONTENLISTE
--------------------------------------------------------
CREATE OR REPLACE TYPE "KONTENLISTE" as table of REF Konto_t;
\end{lstlisting}
\begin{lstlisting}
/
--------------------------------------------------------
--  DDL for Type KONTENLISTE
--------------------------------------------------------
CREATE OR REPLACE TYPE "KONTENLISTE" as table of REF Konto_t;
\end{lstlisting}
\begin{lstlisting}
/
--------------------------------------------------------
--  DDL for Type KONTO_T
--------------------------------------------------------
CREATE OR REPLACE TYPE "KONTO_T" AS OBJECT(
KNR INTEGER,
Kontostand  NUMBER(10,2),
Typ INTEGER(1)) NOT FINAL;
\end{lstlisting}
\begin{lstlisting}
/
--------------------------------------------------------
--  DDL for Type PERSON_T
--------------------------------------------------------
CREATE OR REPLACE TYPE "PERSON_T" AS OBJECT(
Kd_NR INTEGER,
Nachname  VARCHAR2(20),
Vorname   VARCHAR2(20),
Adresse REF  Adresse_t,
Status  INTEGER(1),
Konten  Kontenliste) NOT FINAL;
\end{lstlisting}
\begin{lstlisting}
/
--------------------------------------------------------
--  DDL for Type ZWEIGSTELLE_T
--------------------------------------------------------
CREATE OR REPLACE TYPE "ZWEIGSTELLE_T" AS OBJECT(
ZName VARCHAR2(20),
Adresse REF Adresse_t,
Leiter   Integer,
Konten  Kontenliste) NOT FINAL;
\end{lstlisting}
\begin{lstlisting}
/
--------------------------------------------------------
--  DDL for Table ADRESSEN
--------------------------------------------------------
CREATE TABLE "ADRESSEN" OF "ADRESSE_T" 

--------------------------------------------------------
--  DDL for Table KONTEN
--------------------------------------------------------
CREATE TABLE "KONTEN" OF "KONTO_T" 

--------------------------------------------------------
--  DDL for Table KUNDEN
--------------------------------------------------------
CREATE TABLE "KUNDEN" OF "PERSON_T" 
NESTED TABLE "KONTEN" STORE AS "KONTEN_KD"

--------------------------------------------------------
--  DDL for Table ZWEIGSTELLEN
--------------------------------------------------------
CREATE TABLE "ZWEIGSTELLEN" OF "ZWEIGSTELLE_T" 
NESTED TABLE "KONTEN" STORE AS "KONTEN_ZWEIGST"
\end{lstlisting}
\begin{landscape}
\begin{lstlisting}
--------------------------------------------------------
--  Inserts in Table ADRESSEN
--------------------------------------------------------
Insert into ADRESSEN (STRASSE,HNR,PLZ,ORT) values ('Simseestr.','3','80800','Musterhausen');
Insert into ADRESSEN (STRASSE,HNR,PLZ,ORT) values ('Hochstr.','1','80000','Muenchhausen');
Insert into ADRESSEN (STRASSE,HNR,PLZ,ORT) values ('Eschenweg','12','80335','Muenchen');
Insert into ADRESSEN (STRASSE,HNR,PLZ,ORT) values ('Muenchnerstr.','33','80801','Muenchen');
Insert into ADRESSEN (STRASSE,HNR,PLZ,ORT) values ('Schellingstr.','42','53620','Hasenbuettel');
\end{lstlisting}
\begin{lstlisting}
--------------------------------------------------------
--  Inserts in Table KONTEN
--------------------------------------------------------
Insert into KONTEN (KNR,KONTOSTAND,TYP) values ('120768','234,56','0');
Insert into KONTEN (KNR,KONTOSTAND,TYP) values ('348973','12567,56','1');
Insert into KONTEN (KNR,KONTOSTAND,TYP) values ('678453','-456,78','1');
Insert into KONTEN (KNR,KONTOSTAND,TYP) values ('987654','789,65','1');
Insert into KONTEN (KNR,KONTOSTAND,TYP) values ('745363','-23,67','0');
\end{lstlisting}
\begin{lstlisting}
--------------------------------------------------------
--  Inserts in Table KUNDEN
--------------------------------------------------------
Insert into KUNDEN values (PERSON_T('2345','Fach','Hans',(SELECT REF(adr) FROM ADRESSEN adr WHERE adr.STRASSE = 'Muenchnerstr.'),'0',Kontenliste()));
Insert into TABLE (SELECT KONTEN FROM KUNDEN WHERE KD_NR = '2345')(SELECT REF(k) FROM KONTEN k WHERE k.KNR = '120768' OR k.KNR = '348973');
Insert into KUNDEN values (PERSON_T('7654','Meier','Bernd',(SELECT REF(adr) FROM ADRESSEN adr WHERE adr.STRASSE = 'Eschenweg'),'1',Kontenliste((SELECT REF(k) FROM KONTEN k WHERE k.KNR = '987654'))));
Insert into KUNDEN values (PERSON_T('8764','Wiesner','Jan',(SELECT REF(adr) FROM ADRESSEN adr WHERE adr.STRASSE = 'Schellingstr.'),'0',Kontenliste()));
Insert into TABLE (SELECT KONTEN FROM KUNDEN WHERE KD_NR = '2345')(SELECT REF(k) FROM KONTEN k WHERE k.KNR = '348973' OR k.KNR = '678453' OR k.KNR = '745363');
\end{lstlisting}
\begin{lstlisting}
--------------------------------------------------------
--  Inserts in Table ZWEIGSTELLEN
--------------------------------------------------------
Insert into ZWEIGSTELLEN values ('Bachdorf',(SELECT REF(adr) FROM ADRESSEN adr WHERE adr.strasse = 'Hochstr.'),'1768',Kontenliste());
Insert into TABLE (SELECT KONTEN FROM ZWEIGSTELLEN WHERE ZNAME = 'Bachdorf')(SELECT REF(k) FROM KONTEN k WHERE k.KNR = '120768' OR k.KNR = '348973' OR k.KNR = '678453');
Insert into ZWEIGSTELLEN values ('Riedering',(SELECT REF(adr) FROM ADRESSEN adr WHERE adr.strasse = 'Simseestr.'),'9823',Kontenliste());
Insert into TABLE (SELECT KONTEN FROM ZWEIGSTELLEN WHERE ZNAME = 'Riedering')(SELECT REF(k) FROM KONTEN k WHERE k.KNR = '745363' OR k.KNR = '987654');
\end{lstlisting}
\begin{lstlisting}
-------------------------------------------------------------------------------------------------
--  SELECT Statement gibt "alle Kontonummern mit Konostand, Art und Adresse der Zweigstelle" aus
-------------------------------------------------------------------------------------------------
SELECT z.ZNAME AS Zweigstellenname, DEREF(z.ADRESSE).STRASSE AS Strasse, DEREF(z.ADRESSE).HNR AS Hausnummer, DEREF(z.ADRESSE).PLZ AS PLZ, DEREF(z.ADRESSE).ORT AS ORT, DEREF(k.COLUMN_VALUE).KNR AS Kontonummer, DEREF(k.COLUMN_VALUE).KONTOSTAND AS Kontonstand, DEREF(k.COLUMN_VALUE).TYP AS Typ FROM ZWEIGSTELLEN z, TABLE(z.KONTEN) k;

--> Result
\end{lstlisting}

\begin{tabular}{ l l r r l r r r }
ZWEIGSTELLENNAME & STRASSE & HAUSNUMMER & PLZ & ORT & KONTONUMMER & KONTOSTAND & TYP \\
\hline
Bachdorf   & Hochstr.   & 1  &  80000&  Muenchhausen    & 120768     & 234,56   &    0 \\
Bachdorf   &        Hochstr. &  1  & 80000& Muenchhausen  &   348973   & 12567,56  &     1 \\
Bachdorf   &        Hochstr.  & 1  & 80000 & Muenchhausen  &   678453   &  -456,78  &     1 \\
Riedering  &        Simseestr.& 3  & 80800 & Musterhausen  &   987654   &   789,65   &    1 \\
Riedering  &        Simseestr. & 3 &  80800 & Musterhausen  &   745363   &   -23,67  &     0 \\
\end{tabular}


\begin{lstlisting}

---------------------------------------------------------------------------------
--  SELECT Statement gibt "alle Kontonummern mit Adresse aller Kontoinhaber" aus
---------------------------------------------------------------------------------
SELECT kd.VORNAME AS Vorname, kd.NACHNAME AS Nachname, DEREF(k.COLUMN_VALUE).KNR AS Kontonummer, DEREF(kd.ADRESSE).STRASSE AS Strasse, DEREF(kd.ADRESSE).HNR AS Hausnummer, DEREF(kd.ADRESSE).PLZ AS PLZ, DEREF(kd.ADRESSE).Ort AS ORT FROM KUNDEN kd, TABLE(kd.KONTEN) k;

--> Result
\end{lstlisting}

\begin{tabular}{ l l r l r r l }
VORNAME& NACHNAME &  KONTONUMMER  & STRASSE   &   HAUSNUMMER&  PLZ & ORT \\           
\hline
Bernd &  Meier &  987654& Eschenweg    & 12   &   80335 & Muenchen \\              
Hans &   Fach  &  348973& Muenchnerstr. & 33  &    80801&  Muenchen  \\            
Hans  &  Fach  &  120768& Muenchnerstr.& 33   &   80801 & Muenchen     \\         
Jan  &   Wiesner& 745363& Schellingstr.& 42    &  53620& Hasenbuettel   \\       
Jan  &   Wiesner& 348973& Schellingstr.& 42   &   53620& Hasenbuettel     \\     
Jan   &  Wiesner& 678453& Schellingstr. & 42  &    53620 & Hasenbuettel \\
\end{tabular}
\begin{lstlisting}
--------------------------------------------------------
--  SELFJOIN
--------------------------------------------------------
select kd1.vorname as VORNAME, kd1.nachname as NACHNAME, kd2.vorname, kd2.nachname, deref(k1.COLUMN_VALUE).KNR FROM Kunden kd1, kunden kd2, TABLE(kd1.konten) k1, TABLE(kd2.konten) k2 WHERE deref(k1.COLUMN_VALUE).KNR = deref(k2.COLUMN_VALUE).KNR AND kd1.nachname <> kd2.nachname;
\end{lstlisting}
\end{landscape}

\newpage
\subsection*{2. Möglichkeit}
\begin{lstlisting}
--------------------------------------------------------
--  DDL for Type ADRESSE_T2
--------------------------------------------------------
CREATE OR REPLACE TYPE "ADRESSE_T2" AS OBJECT(
Strasse  VARCHAR2(30),
HNR INTEGER,
PLZ INTEGER,
Ort VARCHAR2(20)) NOT FINAL;
\end{lstlisting}
\begin{lstlisting}
/
--------------------------------------------------------
--  DDL for Type PERSON_T2
--------------------------------------------------------
CREATE OR REPLACE TYPE "PERSON_T2" AS OBJECT(
Kd_NR INTEGER,
Nachname  VARCHAR2(20),
Vorname   VARCHAR2(20),
Adresse REF  Adresse_t2,
Status  INTEGER(1)) NOT FINAL;
\end{lstlisting}
\begin{lstlisting}
/
--------------------------------------------------------
--  DDL for Type KUNDENLISTE2
--------------------------------------------------------
CREATE OR REPLACE TYPE "KUNDENLISTE2" as table of REF PERSON_t2;%
\end{lstlisting}
\begin{lstlisting}
/
--------------------------------------------------------
--  DDL for Type KONTO_T2
--------------------------------------------------------
CREATE OR REPLACE TYPE "KONTO_T2" AS OBJECT(
KNR INTEGER,
Kontostand  NUMBER(10,2),
Typ INTEGER(1),
Kunden2 Kundenliste2) NOT FINAL;
\end{lstlisting}
\begin{lstlisting}
/
--------------------------------------------------------
--  DDL for Type KONTENLISTE2
--------------------------------------------------------
CREATE OR REPLACE TYPE "KONTENLISTE2" as table of Konto_t2;
\end{lstlisting}
\begin{lstlisting}
/
--------------------------------------------------------
--  DDL for Type ZWEIGSTELLE_T2
--------------------------------------------------------
CREATE OR REPLACE TYPE "ZWEIGSTELLE_T2" AS OBJECT(
ZName VARCHAR2(20),
Adresse2 REF Adresse_t2,
Leiter   Integer,
Konten2  Kontenliste2) NOT FINAL;
\end{lstlisting}
\begin{lstlisting}
/
--------------------------------------------------------
--  DDL for Table ADRESSEN2
--------------------------------------------------------
CREATE TABLE "ADRESSEN2" OF "ADRESSE_T2" 
\end{lstlisting}

\begin{lstlisting}
--------------------------------------------------------
--  DDL for Table KUNDEN2
--------------------------------------------------------
CREATE TABLE "KUNDEN2" OF "PERSON_T2" 
\end{lstlisting}
\begin{lstlisting}
--------------------------------------------------------
--  DDL for Table ZWEIGSTELLEN
--------------------------------------------------------
CREATE TABLE "ZWEIGSTELLEN2" OF "ZWEIGSTELLE_T2" 
NESTED TABLE "KONTEN2" STORE AS "KONTEN_ZWEIGST2" (NESTED TABLE "KUNDEN2" STORE AS "KUNDEN_ZWEIGST2");
\end{lstlisting}
\begin{landscape}
\begin{lstlisting}
--------------------------------------------------------
--  Inserts in Table ADRESSEN2
--------------------------------------------------------
Insert into ADRESSEN2 (STRASSE,HNR,PLZ,ORT) values ('Simseestr.','3','80800','Musterhausen');
Insert into ADRESSEN2 (STRASSE,HNR,PLZ,ORT) values ('Hochstr.','1','80000','Muenchhausen');
Insert into ADRESSEN2 (STRASSE,HNR,PLZ,ORT) values ('Eschenweg','12','80335','Muenchen');
Insert into ADRESSEN2 (STRASSE,HNR,PLZ,ORT) values ('Muenchnerstr.','33','80801','Muenchen');
Insert into ADRESSEN2 (STRASSE,HNR,PLZ,ORT) values ('Schellingstr.','42','53620','Hasenbuettel');
\end{lstlisting}
\begin{lstlisting}
--------------------------------------------------------
--  Inserts in Table KUNDEN2
--------------------------------------------------------
Insert into KUNDEN2 values ('2345','Fach','Hans',(SELECT REF(adr) FROM ADRESSEN2 adr WHERE adr.STRASSE = 'Muenchnerstr.'),'0');
Insert into KUNDEN2 values ('7654','Meier','Bernd',(SELECT REF(adr) FROM ADRESSEN2 adr WHERE adr.STRASSE = 'Eschenweg'),'1');
Insert into KUNDEN2 values ('8764','Wiesner','Jan',(SELECT REF(adr) FROM ADRESSEN2 adr WHERE adr.STRASSE = 'Schellingstr.'),'0');
\end{lstlisting}
\begin{lstlisting}
--------------------------------------------------------
--  Inserts in Table ZWEIGSTELLEN
--------------------------------------------------------
Insert into ZWEIGSTELLEN2 values ('Bachdorf',(SELECT REF(adr) FROM ADRESSEN2 adr WHERE adr.STRASSE = 'Hochstr.'),'1768',Kontenliste2(KONTO_T2('120768','234,56','0',Kundenliste2()),KONTO_T2('348973','12567,56','1',Kundenliste2()),KONTO_T2('678453','-456,78','1',Kundenliste2())));
--> Bemerkung: INSERT into table (SELECT z.KONTEN2 FROM ZWEIGSTELLEN2 z WHERE z.ZNAME ='BACHDORF') values('678453','-456,78','1',Kundenliste2()); //Einfuegen in die Nested Table Konten
Insert into ZWEIGSTELLEN2 values ('Riedering',(SELECT REF(adr) FROM ADRESSEN2 adr WHERE adr.STRASSE = 'Simseestr.'),'9823',Kontenliste2(KONTO_T2('987654','789,65','1',Kundenliste2((SELECT REF(kd) FROM KUNDEN2 kd WHERE kd.KD_NR = '8764'))),KONTO_T2('745363','-23,67','0',Kundenliste2((SELECT REF(kd) FROM KUNDEN2 kd WHERE kd.KD_NR = '7654')))));

-- befuellen der jeweiligen NESTED TABLE Kundenliste2
Insert into TABLE (SELECT k.KUNDEN2 FROM ZWEIGSTELLEN2 z, TABLE(z.KONTEN2) k WHERE z.ZNAME = 'Bachdorf' AND k.KNR = '120768')(SELECT REF(kd) FROM KUNDEN2 kd WHERE kd.KD_NR = '2345');
Insert into TABLE (SELECT k.KUNDEN2 FROM ZWEIGSTELLEN2 z, TABLE(z.KONTEN2) k WHERE z.ZNAME = 'Bachdorf' AND k.KNR = '348973')(SELECT REF(kd) FROM KUNDEN2 kd WHERE kd.KD_NR = '2345' OR kd.KD_NR = '8764');
Insert into TABLE (SELECT k.KUNDEN2 FROM ZWEIGSTELLEN2 z, TABLE(z.KONTEN2) k WHERE z.ZNAME = 'Bachdorf' AND k.KNR = '678453')(SELECT REF(kd) FROM KUNDEN2 kd WHERE kd.KD_NR = '8764');

-- Alternatvie wenn SELECT nicht direkt in Kundenliste2()
--> Insert into TABLE (SELECT k.KUNDEN2 FROM ZWEIGSTELLEN2 z, TABLE(z.KONTEN2) k WHERE z.ZNAME = 'Riedering' AND k.KNR = '987654')(SELECT REF(kd) FROM KUNDEN2 kd WHERE kd.KD_NR = '7654');
--> Insert into TABLE (SELECT k.KUNDEN2 FROM ZWEIGSTELLEN2 z, TABLE(z.KONTEN2) k WHERE z.ZNAME = 'Riedering' AND k.KNR = '745363')(SELECT REF(kd) FROM KUNDEN2 kd WHERE kd.KD_NR = '8764');

\end{lstlisting}
\begin{lstlisting}
-------------------------------------------------------------------------------------------------
--  SELECT Statement gibt "alle Kontonummern mit Konostand, Art und Adresse der Zweigstelle" aus
-------------------------------------------------------------------------------------------------
SELECT z.ZNAME AS Filiale, DEREF(z.ADRESSE2).STRASSE AS Strasse, DEREF(z.ADRESSE2).HNR AS Hausnummer, DEREF(z.ADRESSE2).PLZ AS PLZ, DEREF(z.ADRESSE2).ORT AS Ort ,k.KNR AS Kontonummer, k.KONTOSTAND AS Kontostand, k.TYP AS Typ FROM ZWEIGSTELLEN2 z, TABLE(z.KONTEN2) k;

--> Result:
\end{lstlisting}
\begin{tabular}{ l l r r l r r r }
FILIALE     &         STRASSE             &           HAUSNUMMER    &    PLZ & ORT            &      KONTONUMMER & KONTOSTAND    &    TYP \\
\hline
Bachdorf         &    Hochstr.   &                             1   &   80000 & Muenchhausen  &             120768  &   234,56   &       0 \\
Bachdorf       &      Hochstr.     &                           1   &   80000 & Muenchhausen   &            348973 &  12567,56      &    1 \\
Bachdorf    &         Hochstr.      &                          1  &    80000&  Muenchhausen   &            678453  &  -456,78    &      1 \\
Riedering    &        Simseestr.    &                          3   &   80800& Musterhausen    &          987654   &  789,65    &      1 \\
Riedering    &        Simseestr.    &                          3  &    80800& Musterhausen    &          745363  &   -23,67    &      0 \\
\end{tabular}

\begin{lstlisting}

---------------------------------------------------------------------------------
--  SELECT Statement gibt "alle Kontonummern mit Adresse aller Kontoinhaber" aus
---------------------------------------------------------------------------------
SELECT DEREF(kd.COLUMN_VALUE).VORNAME AS Vorname, DEREF(kd.COLUMN_VALUE).NACHNAME AS Nachname, DEREF(DEREF(kd.COLUMN_VALUE).ADRESSE).STRASSE AS Strasse, DEREF(DEREF(kd.COLUMN_VALUE).ADRESSE).HNR AS Hausnummer, DEREF(DEREF(kd.COLUMN_VALUE).ADRESSE).PLZ AS PLZ, DEREF(DEREF(kd.COLUMN_VALUE).ADRESSE).ORT AS Ort, k.KNR AS Kontonummer, k.Kontostand AS Kontostand FROM ZWEIGSTELLEN2 z, TABLE(z.KONTEN2) k, TABLE(k.KUNDEN2) kd;

--> Result:
\end{lstlisting}
\begin{tabular}{ l l l r r l r r }
VORNAME      &        NACHNAME     &        STRASSE                &        HAUSNUMMER    &    PLZ & ORT          &        KONTONUMMER & KONTOSTAND \\
\hline
Hans        &         Fach        &         Muenchnerstr.             &              33   &   80801 & Muenchen   &                120768  &   234,56 \\
Hans       &          Fach      &           Muenchnerstr.       &                    33    &  80801  & Muenchen      &             348973 &   12567,56 \\
Jan          &        Wiesner       &       Schellingstr.            &              42    &  53620 & Hasenbuettel    &           348973  & 12567,56 \\
Jan        &          Wiesner    &          Schellingstr.   &                       42  &    53620 & Hasenbuettel   &            678453  &  -456,78 \\
Jan           &       Wiesner      &        Schellingstr.      &                    42   &   53620 & Hasenbuettel       &        987654  &   789,65 \\
Bernd      &          Meier     &           Eschenweg   &                           12    &  80335 & Muenchen              &     745363   &  -23,67 \\
\end{tabular}

\end{landscape}
		
\end{document}